\documentclass{szablon}

\usepackage{graphicx}
\usepackage{listings}
\usepackage{xcolor}
\usepackage{hyperref}
\usepackage{float}

% Konfiguracja listingów kodu Java
\lstset{
    language=Java,
    basicstyle=\ttfamily\footnotesize,
    keywordstyle=\color{blue}\bfseries,
    stringstyle=\color{red},
    commentstyle=\color{gray}\itshape,
    breaklines=true,
    frame=single,
    numbers=left,
    numberstyle=\tiny\color{gray},
    tabsize=4,
    showstringspaces=false
}

\title{Techniki programowania III \\PROJEKT}
\author{Dawid Sułek, Dominik Rodziewicz \\ Uniwersytet Zielonogórski}

\begin{document}

\chapter{Sieciowa Tablica Ogłoszeń}

\section{Wstęp}
Projekt „Sieciowa Tablica Ogłoszeń" to aplikacja klient-serwer napisana w języku Java z wykorzystaniem JavaFX do interfejsu graficznego oraz JDBC do komunikacji z bazą danych MySQL. System umożliwia użytkownikom przeglądanie, dodawanie, edytowanie i usuwanie ogłoszeń w czasie rzeczywistym, z pełnym wsparciem dla wielu jednoczesnych połączeń klientów.

Główne cele projektu:
\begin{itemize}
    \item Implementacja architektury klient-serwer z komunikacją przez sockety
    \item Obsługa wielu klientów jednocześnie (wielowątkowość)
    \item Przechowywanie danych w relacyjnej bazie MySQL
    \item Nowoczesny interfejs graficzny w JavaFX z motywem „pergaminowym"
    \item System autoryzacji z rozróżnieniem ról (użytkownik/administrator)
    \item Powiadomienia w czasie rzeczywistym dla wszystkich podłączonych klientów
\end{itemize}

\subsection{Opis projektu}
Aplikacja realizuje następujące funkcjonalności:

\textbf{Dla wszystkich użytkowników:}
\begin{enumerate}
    \item Rejestracja nowego konta z walidacją danych
    \item Logowanie z hashowaniem hasła SHA-256
    \item Przeglądanie tablicy ogłoszeń
    \item Filtrowanie ogłoszeń według kategorii, autora i tekstu
    \item Sortowanie ogłoszeń (data, tytuł, popularność)
    \item Wyświetlanie szczegółów ogłoszenia (zwiększa licznik wyświetleń)
\end{enumerate}

\textbf{Dla zalogowanych użytkowników:}
\begin{enumerate}
    \item Dodawanie nowych ogłoszeń z kategoriami
    \item Edycja własnych ogłoszeń
    \item Usuwanie własnych ogłoszeń
    \item Zgłaszanie nieodpowiednich ogłoszeń
    \item Odbieranie powiadomień o zmianach w czasie rzeczywistym
\end{enumerate}

\textbf{Dla administratorów:}
\begin{enumerate}
    \item Usuwanie dowolnych ogłoszeń
    \item Przeglądanie zgłoszonych ogłoszeń
    \item Generowanie raportów z tablicy ogłoszeń
\end{enumerate}

\subsection{Dokumentacja UML}

\begin{figure}[H]
    \centering
    \includegraphics[width=0.9\textwidth]{diagram_klas.png}
    \caption{Diagram klas systemu Sieciowa Tablica Ogłoszeń}
    \label{fig:diagram_klas}
\end{figure}

\begin{figure}[H]
    \centering
    \includegraphics[width=0.9\textwidth]{diagram_przypadkow_uzycia.png}
    \caption{Diagram przypadków użycia systemu Sieciowa Tablica Ogłoszeń}
    \label{fig:diagram_przypadkow_uzycia}
\end{figure}

\subsection{Struktura projektu}

\textbf{Hierarchia pakietów:}
\begin{verbatim}
src/main/java/pl/tablicaogloszen/
├── wspolne/          # Klasy współdzielone (DTO, enum)
│   ├── OgloszenieDTO.java
│   ├── UzytkownikDTO.java
│   ├── FiltrDTO.java
│   ├── Zadanie.java
│   ├── Odpowiedz.java
│   ├── TypZadania.java
│   └── StatusOdpowiedzi.java
├── serwer/           # Logika serwera
│   ├── Serwer.java
│   ├── ObslugaKlienta.java
│   ├── PolaczenieBazy.java (Singleton)
│   ├── InicjalizatorBazy.java
│   ├── OgloszenieDAO.java
│   ├── UzytkownikDAO.java
│   └── Bezpieczenstwo.java
└── klient/           # Aplikacja klienta
    ├── AplikacjaKlienta.java
    ├── KlientSieciowy.java (Singleton)
    ├── KontrolerLogowania.java
    ├── KontrolerRejestracji.java
    ├── KontrolerTablicy.java
    └── Sesja.java
\end{verbatim}

\textbf{Wzorce projektowe:}
\begin{itemize}
    \item \textbf{Singleton} -- \texttt{PolaczenieBazy}, \texttt{KlientSieciowy}
    \item \textbf{DAO (Data Access Object)} -- \texttt{OgloszenieDAO}, \texttt{UzytkownikDAO}
    \item \textbf{DTO (Data Transfer Object)} -- wszystkie klasy w pakiecie \texttt{wspolne}
    \item \textbf{MVC} -- kontrolery FXML + widoki FXML + modele DTO
\end{itemize}

\textbf{Implementacja interfejsu Serializable:}
Wszystkie klasy DTO implementują \texttt{java.io.Serializable}, co umożliwia przesyłanie obiektów przez strumienie obiektowe (\texttt{ObjectOutputStream}/\texttt{ObjectInputStream}).

\textbf{Wielowątkowość:}
\begin{itemize}
    \item Serwer używa \texttt{ExecutorService} z \texttt{newCachedThreadPool()} do obsługi wielu klientów
    \item Każdy klient jest obsługiwany w osobnym wątku (\texttt{ObslugaKlienta implements Runnable})
    \item Klient JavaFX używa \texttt{Platform.runLater()} do aktualizacji UI z wątków roboczych
\end{itemize}

\subsection{Interfejs użytkownika}

Interfejs graficzny został zbudowany w JavaFX z wykorzystaniem FXML i arkuszy stylów CSS. Zastosowano motyw „pergaminowy" nawiązujący do stylu fantasy.

\textbf{Główne ekrany:}
\begin{enumerate}
    \item \textbf{Ekran logowania} (\texttt{logowanie.fxml}) -- pola login/hasło, przyciski „Zaloguj" i „Rejestracja"
    \item \textbf{Ekran rejestracji} (\texttt{rejestracja.fxml}) -- pola login/hasło/powtórz hasło z walidacją
    \item \textbf{Tablica ogłoszeń} (\texttt{tablica.fxml}) -- główny widok z kartami ogłoszeń
\end{enumerate}

\textbf{Elementy ergonomii:}
\begin{itemize}
    \item Filtry umieszczone w górnym pasku (kategoria, autor, tekst, sortowanie)
    \item Karty ogłoszeń z losową rotacją dla efektu „przypięcia do tablicy"
    \item Przyciski akcji (edycja, usuwanie, zgłaszanie) na każdej karcie
    \item Dialogi modalne do dodawania/edycji ogłoszeń
    \item Powiadomienia Alert dla informacji zwrotnych
    \item Tooltips na przyciskach dla lepszej dostępności
\end{itemize}

\textbf{Obsługa zdarzeń:}
\begin{itemize}
    \item \texttt{onAction} -- przyciski (filtrowanie, dodawanie, edycja)
    \item \texttt{onMouseClicked} -- kliknięcie karty otwiera szczegóły
    \item \texttt{Platform.runLater()} -- aktualizacja UI z wątków sieciowych
\end{itemize}

\subsection{Obsługa wyjątków i walidacja danych}

\textbf{Obsługa wyjątków:}
\begin{lstlisting}
// Przykład obsługi wyjątków w ObslugaKlienta.java
try {
    while (true) {
        Object obiekt = wejscie.readObject();
        if (obiekt instanceof Zadanie) {
            obslozZadanie((Zadanie) obiekt);
        }
    }
} catch (EOFException e) {
    // Klient się rozłączył - normalne zakończenie
} catch (IOException | ClassNotFoundException e) {
    e.printStackTrace();
} finally {
    zamknijPolaczenie();
    Serwer.usunKlienta(this);
}
\end{lstlisting}

\textbf{Typy obsługiwanych wyjątków:}
\begin{itemize}
    \item \texttt{SQLException} -- błędy bazy danych (obsługiwane w DAO)
    \item \texttt{IOException} -- błędy komunikacji sieciowej
    \item \texttt{EOFException} -- rozłączenie klienta
    \item \texttt{ClassNotFoundException} -- błąd deserializacji
    \item \texttt{NoSuchAlgorithmException} -- błąd hashowania (SHA-256)
\end{itemize}

\textbf{Walidacja danych:}
\begin{lstlisting}
// Walidacja przy rejestracji (KontrolerRejestracji.java)
if (login.isEmpty() || haslo.isEmpty()) {
    utworzAlert(Alert.AlertType.WARNING, 
        "Uzupelnij wszystkie pola!");
    return;
}
if (!haslo.equals(powtorzHaslo)) {
    utworzAlert(Alert.AlertType.WARNING, 
        "Hasla sie nie zgadzaja!");
    return;
}
\end{lstlisting}

\textbf{Walidacja po stronie serwera:}
\begin{itemize}
    \item Sprawdzanie uprawnień przed edycją/usuwaniem ogłoszeń
    \item Weryfikacja roli administratora dla funkcji administracyjnych
    \item Sprawdzanie czy użytkownik jest zalogowany
\end{itemize}

\subsection{Realizacja zadania}

\textbf{1. Hashowanie hasła (SHA-256):}
\begin{lstlisting}
// Bezpieczenstwo.java
public static String hashuj(String haslo) {
    try {
        MessageDigest md = MessageDigest.getInstance("SHA-256");
        byte[] hash = md.digest(haslo.getBytes(StandardCharsets.UTF_8));
        StringBuilder sb = new StringBuilder();
        for (byte b : hash) {
            sb.append(String.format("%02x", b));
        }
        return sb.toString();
    } catch (NoSuchAlgorithmException e) {
        throw new RuntimeException("SHA-256 niedostepny", e);
    }
}
\end{lstlisting}

\textbf{2. Komunikacja klient-serwer:}
\begin{lstlisting}
// KlientSieciowy.java - wysyłanie żądania
public synchronized Odpowiedz wyslijISprawdz(Zadanie zadanie) {
    try {
        wyjscie.writeObject(zadanie);
        wyjscie.flush();
        return oczekujacaOdpowiedz.poll(5, TimeUnit.SECONDS);
    } catch (Exception e) {
        e.printStackTrace();
        return null;
    }
}
\end{lstlisting}

\textbf{3. Powiadomienia real-time:}
\begin{lstlisting}
// Serwer.java - powiadamianie wszystkich klientów
public static void powiadomWszystkich() {
    Odpowiedz odswiez = new Odpowiedz(
        StatusOdpowiedzi.ODSWIEZ, null, "Odswiezenie");
    synchronized (klienci) {
        for (ObslugaKlienta k : klienci) {
            k.wyslij(odswiez);
        }
    }
}
\end{lstlisting}

\textbf{4. Filtrowanie i sortowanie ogłoszeń:}
\begin{lstlisting}
// OgloszenieDAO.java - budowanie zapytania SQL
switch (filtr.getSortowanie()) {
    case "DATA_DESC": sql.append(" ORDER BY o.data_dodania DESC"); break;
    case "TYTUL_ASC": sql.append(" ORDER BY o.tytul ASC"); break;
    case "POPULARNOSC_DESC": sql.append(" ORDER BY o.wyswietlenia DESC"); break;
    // ...
}
\end{lstlisting}

\textbf{5. System zgłaszania ogłoszeń:}
\begin{lstlisting}
// OgloszenieDAO.java
public void zglosOgloszenie(int idOgloszenia) throws SQLException {
    String sql = "UPDATE ogloszenia SET zgloszenia = COALESCE(zgloszenia, 0) + 1 WHERE id = ?";
    try (Connection pol = PolaczenieBazy.pobierzPolaczenie();
         PreparedStatement pstm = pol.prepareStatement(sql)) {
        pstm.setInt(1, idOgloszenia);
        pstm.executeUpdate();
    }
}
\end{lstlisting}

\subsection{Javadoc}

Dokumentacja Javadoc została wygenerowana dla wszystkich klas publicznych projektu. Każda klasa zawiera:
\begin{itemize}
    \item Opis klasy i jej przeznaczenia
    \item Dokumentację konstruktorów i metod publicznych
    \item Tagi \texttt{@author Dawid Sułek, Dominik Rodziewicz}
    \item Tagi \texttt{@param}, \texttt{@return}, \texttt{@throws} gdzie stosowne
\end{itemize}

Przykład dokumentacji:
\begin{lstlisting}
/**
 * Klasa odpowiedzialna za bezpieczne hashowanie hasel.
 * Uzywa algorytmu SHA-256 do jednokierunkowego szyfrowania.
 * 
 * @author Dawid Sulek, Dominik Rodziewicz
 */
public class Bezpieczenstwo {
    /**
     * Hashuje haslo algorytmem SHA-256 i zwraca jego hex.
     * 
     * @param haslo haslo w postaci tekstowej
     * @return hash SHA-256 w formacie szesnastkowym (64 znaki)
     */
    public static String hashuj(String haslo) { ... }
}
\end{lstlisting}

\section{Wnioski}

Realizacja projektu „Sieciowa Tablica Ogłoszeń" pozwoliła na praktyczne zastosowanie:

\begin{enumerate}
    \item \textbf{Programowania obiektowego} -- enkapsulacja, dziedziczenie interfejsów (\texttt{Serializable}, \texttt{Runnable}), polimorfizm
    \item \textbf{Wzorców projektowych} -- Singleton, DAO, DTO, MVC
    \item \textbf{Programowania współbieżnego} -- ExecutorService, synchronizacja, Platform.runLater()
    \item \textbf{Komunikacji sieciowej} -- sockety TCP, strumienie obiektowe
    \item \textbf{Baz danych} -- JDBC, PreparedStatement, transakcje
    \item \textbf{Interfejsów graficznych} -- JavaFX, FXML, CSS, obsługa zdarzeń
    \item \textbf{Bezpieczeństwa} -- hashowanie haseł, autoryzacja, walidacja danych
\end{enumerate}

Projekt może być rozszerzony o:
\begin{itemize}
    \item Szyfrowanie komunikacji (SSL/TLS)
    \item System kategorii definiowanych przez administratora
    \item Załączniki graficzne do ogłoszeń
    \item Powiadomienia push na urządzenia mobilne
\end{itemize}


\section{Sposób oceny projektu}	
\begin{tabular}{|c|c|c|}
	\hline
	Element & Punkty & Uzyskane \\
	\hline
	UML (diagramy klas, przypadków użycia) & 10 & \\
	\hline
	Struktura projektu + OOP (dziedziczenie, interfejsy, pakiety) & 20 & \\
	\hline
	Implementacja funkcjonalności & 30  &\\
	\hline
	GUI (ergonomia, poprawność, zdarzenia) & 20  &\\
	\hline
	Obsługa wyjątków i walidacja danych & 10  &\\
	\hline
	Dokumentacja Javadoc & 5 & \\
	\hline
	Prezentacja projektu & 5  &\\
	\hline
    RAZEM & 100  &\\
    \hline
\end{tabular}

\subsection{Skala ocen}
\begin{itemize}
\item 90–100 pkt → 5.0
\item 80–89 pkt → 4.5
\item 70–79 pkt → 4.0
\item 60–69 pkt → 3.5
\item 50–59 pkt → 3.0
\item poniżej 50 pkt → 2.0 (niezaliczone)
\end{itemize}

\end{document}
